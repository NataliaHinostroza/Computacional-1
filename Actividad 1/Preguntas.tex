\documentclass[11pt]{article}
\usepackage[spanish]{babel}
\usepackage[utf8]{inputenc}
\usepackage[T1]{fontenc}
\usepackage{graphicx}
\topmargin=-2.5cm
\textheight=24cm
\textwidth=16cm  
\oddsidemargin=0.45cm  
\setlength{\parindent}{0cm}
\renewcommand{\baselinestretch}{1.1}
\title{Preguntas de Reflexión}
\author{Hinostroza Moya Natalia}
\date{30 de enero de 2016}

%=========================================================================
% INICIO
%=========================================================================
\begin{document}
\textbf{\section*{\Large Preguntas de Reflexión}}
\begin{enumerate}
\item \large \textbf{¿Cual es tu primera impresión de uso de LaTeX?}\\
No es tan complicado aprender el lenguaje de este programa

\item\large \textbf{¿Qué aspectos te gustaron más?}\\
La formalidad con la que se ven los trabajos. 

\item\large \textbf{¿Qué no pudiste hacer en LaTeX?}\\
Al principio no sabía cómo agregar algunos caracteres al texto, pero ya investigue que hay algunos caracteres especiales que tienen su propio comando para poder ser agregados. 

\item\large \textbf{En tu experiencia, comparado con otros editores, ¿cómo se compara LaTeX}?\\
Es más profesional en cuanto a la presentación. 

\item\large \textbf{¿Qué es lo que mas te llamó la atención en el desarrollo de esta actividad?}\\
Que durante todo el tiempo en el que redacté el trabajo estube aprendiendo cosas nuevas sobre LaTeX. Cada vez surgían dudas de cómo hacer o modificar algo en el texto. 

\item\large \textbf{¿Qué cambiarías en esta actividad?}\\
Nada, es una actividad bastante completa encuanto a aprender a utilizar el editor de textos y practicar nuestra redacción. 

\item\large \textbf{¿Que consideras que falta en esta actividad?}
No considero que haga falta algo. Las indicaciones son bastante claras. 

\item\large \textbf{¿Puedes compartir alguna referencia nueva que consideras util y no se haya contemplado?}
No tengo ninguna referencian pero sería bueno agregar algún manual sobre el uso de LaTeX. No encontre ninguno que me funcionara completamente. 

\item\large \textbf{¿Algún comentario adicional que desees compartir?}
Es una buena actividad. Siempre me parecio muy complicado el uso de LaTeX, y ahora que aprendí a utilizarlo me siento muy satisfecha por que lo aprendí por mi cuenta. 

\vspace*{3cm}
\large Alumna: Hinostroza Moya Natalia\\
\large Prof. Carlos Lizárraga 

\end{enumerate}

\end{document}
