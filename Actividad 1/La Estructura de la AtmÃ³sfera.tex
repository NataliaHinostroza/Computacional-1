\documentclass[11pt]{article}
\usepackage[spanish]{babel}
\usepackage[utf8]{inputenc}
\usepackage[T1]{fontenc}
\usepackage{graphicx}
\topmargin=-1.2cm
\textheight=22cm
\textwidth=16cm  
\oddsidemargin=0.45cm  
\setlength{\parindent}{0cm}
\renewcommand{\baselinestretch}{1.1}
\graphicspath{{muajaja/}}
\title{La Estructura de la Atmósfera}
\author{Hinostroza Moya Natalia}
\date{30 de enero de 2017}

\begin{document}
%=======================================================================================
% PORTADA
%=======================================================================================
\begin{titlepage}
\begin{center}
\includegraphics[scale=0.35]{aescudo.png}
\end{center}
\vspace*{0.02in}
\begin{center}

\rmfamily\textbf{\LARGE UNIVERSIDAD DE SONORA}\\
\vspace*{1.02in}
{\Large División de Ciencias Exactas y Naturales}\\
{\Large Departamento de Física}\\

\vspace*{0.99in}
\rule{99mm}{0.1mm}\\
\vspace*{0.15in}
\textbf{\LARGE La Estructura de la Atmósfera}\\
\vspace*{0.001in}
\rule{99mm}{0.1mm}

\vspace*{1in}
\normalsize{Autor:}\
\normalsize{Natalia Hinostroza Moya}\\
\vspace*{0.3mm}
\normalsize Prof. Carlos Lizárraga\\
\vspace*{1cm}
\normalsize 31 de enero de 2017



\end{center}
\end{titlepage}

%=======================================================================================
% DESARROLLO
%=======================================================================================

\textbf{\section{\Large Introducción}}
\large Alguna vez te has preguntado qué es lo que nos rodea o lo que respiramos. Ya hace mucho tiempo se vienen estudiando las distintas capas que conforman el planeta en el que vivmos, pues es muy importante saber de que está compuesto.\\

Con esto inicio el estudio de la ultima de las capas que cnforma a la tierra, la atmósfera. Esta esta conformada por gases que permiten la vida en la superficie terrestre y al igual que la parte interior de la tierra, la atmósfera tiene sus subcapas.\\

En este breve ensayo podremos encontrar una definición concreta de lo que es la atmosfera y cuales son las consecuencias de su existencia.Así mismo se hablará hacerca de cada una de las regiones que conforman la atmósfera, que las hace distintas una de las otras, las distancias a las que se encuentran y los elementos que las conforman 

\textbf{\section{\Large{La atmósfera terrestre y sus componentes}}}
\large La atmósfera es la capa externa de la tierra, de aproximadamente 10,000km de espesor.  Está compuesta de gases, partículas sólidas y partículas líquidas suspendidas que son atraidas por la gravedad. Es la capa en la que se producen los fenomenos climáticos y meteorológicos, donde se resgula la entrada y salida de energía de la tierra.\\

Actualmente, la atmósfera está compuesta por tres gases que componen un porcentaje de 99.95 del volúmen atmosférico; estos son el nitrógeno, oxígeno y argón.De estos tres gases, el nitródeno y el argón son geoquímicamente inertes, o cual quiere decir que permanecen en la atmósfera sin hacer reacción con algún otro eleménto. Sin embargo, el oxígeno es muy activo  y se destaca por la rapidez con la que el oxígeno libre reacciona con los depósitos existentes en las rocas sedimentarias.\\ 

Los gases restantes componentes del aire, como el dióxido de carbono y el ozono, están presentes en la atmósfera en cantidades muy pequeñas que son expresadas volumetricamente en partes por millon (ppm) o en partes por billon (ppb).\\

\textbf{\section{\Large Capas  de la atmósfera}}

\subsection{Por composición}
La atmósfera está dividida de acuerdo a su composición en dos capas; estas son la homoatmósfera y la heteroatmósfera.\\

\begin{itemize}

\item\textbf{Homoatmosfera:}
Va desde los 0 km hasta los 60 km de altura aproximadamente. Está formada por la mezcla de gases que coforman el aire. En su  mayoria es dinitrógeno, dioxígeno, argón, agua y dióxido de carbono.

\item\textbf{Heteroatmósfera:}
Abarca de los 60 Km hasta el final de la atmósfera. En dicha zona los gases en capas dependiendo al peso del átomo que allí se encuentra. Entre las capas podemos distinguir principalmente la del oxígeno, la capa de helio y por último la capa del hidrógeno.

\end{itemize}

\subsection{Por características físicas}
Otra manera en la que se divide la atmósfera es de acuerdo a la variación de sus características físicas con rspecto a la altura. Dentro de esta divición podemos encontrar cuatro capas distintas, la tropósfera, estratósfera, mesósfera y la ionósfera.

\begin{itemize}

\item\textbf{Tropósfera} (0km - 12km)\textbf{:}
Esta es la capa de la atmósfera en la que se desarrolla la vida y se producen los fenómenos atmosféricos. El final de esta capa se conoce como la Tropopausa.

\item\textbf{Estratósfera} (12km - 45km)\textbf{:}
En esta capa se nota un aumento de la temperatura que es capaz de alcanzar los 100$^{\circ}$C. En ella se encuentra la capa de ozono (la ozonósfera), lo que imposibilita el desarrollo de la vida; pues el ozono es un gas estable que absorve radiaciones UV. La estratósfera termina en la Estratopausa.

\item\textbf{Mesósfera} (40km - 90km)\textbf{:}
La temperatura en esta capa puede llegar hasta los -80$^{\circ}$C. Se termina en la Mesopausa. 

\item\textbf{Ionósfera o Termósfera} (90km - 500km)\textbf{:}
Es llamada ionósfera porque los átomos y moléculas existentes en dicha capa se encuentran en forma de iones, es decir, con carga eléctrica. La temperatura de esta capa  puede aumentar hasta los 1,500$^{\circ}$C; esto se debe a la absorción de la energía de las radiaciones que llegan. En ella se producen la reflexiónn de las ondas de radio y televisión.

\item\textbf{Exósfera:}
Va desde los 500km hasta el final de la atmósfera. Esta capa está casi al vacío y la presencia del aire es muy rar. La partículas en la exósfera se mueven con bastante rapidez ya que están casí al vacío; esto produce que dichas partículas estén a temperaturas muy altas. Sin embargo. nosotros sentiriamos temperaturas muy bajas ya que en la exósfera hay muy pocas partículas. 
\vspace*{0.25cm}
A la altura de esta capa, la luz del sol es muy brillante, por lo que los objetos que son iluminados con ella pueden alcanzar temperaturas extremadamente altas, pero en la sombra pueden estar altamente frios. 

\end{itemize}

\vspace*{0.5cm}
\begin{center}
\includegraphics[scale=0.40]{atmosfera.jpg}
\end{center}

\textbf{\section{\Large Presión atmosférica}}
\large Dentro de la tropósfera, la temperatura se va volviendo menor con la elevación; por ello sabemos que, dentro de la capa en la que se dearrolla la vida, entre mayor sea la altura, menor es la presión. \\

Pero como ya vimos anteriormente, en la estratósfera sucede lo contrario con la temperatura, esto se debe a la concentración de ozono en ella. La temperatura se vuelve cada vez mayor con la elevación, por lo que la presión en está región va aumentando mientras más arriba se meda.\\ 

La presión atmosférica se debe al peso de las capas de aire. Por esto es que a mayor altura hay menor presión, pues hay menos capas de aire que quedan por encima; pero considerando que la densidad del aire en la atmosfera es menor cuando mayor sea la altura, resulta que la presión tambien va disminuyendo exponencialmente con la altura, o volviendo a ser mayor dependiendo de en que región de la atmósfera nos encontremos.
\pagebreak

%============================================================================================
% BIBLIOGRAFIA
%============================================================================================
\textbf{\section{\Large Bibliografía}}
\begin{enumerate}

\item Proyecto Biosfera. (2017). Recursos.cnice.mec.es. Retrieved 30 January 2017, from\\ http://recursos.cnice.mec.es/biosfera/alumno/3ESO/energia\_externa/contenidos6.htm

\item Atmosfera. (2017). www7.uc.cl. Retrieved 30 January 2017, from\\ http://www7.uc.cl/sw\_educ/contam/fratmosf.htm

\item Temperatura en la Exosfera  - Ventanas al Universo. (2017). Windows2universe.org. Retrieved 30 January 2017, from\\ http://www.windows2universe.org/earth/Atmosphere/exosphere\_temperature.html\&lang=sp

\item Structure of the Atmosphere  | Climate Education Modules for K-12. (2017). Climate.ncsu.edu. Retrieved 30 January 2017, from\\ http://climate.ncsu.edu/edu/k12/.AtmStructure

\item Presion atmosferica - Ventanas al Universo. (2017). Windows2universe.org. Retrieved 30 January 2017, from\\ http://www.windows2universe.org/earth/Atmosphere/atm\_press.html\&lang=sp

\end{enumerate}

%============================================================================================
% FIN
%============================================================================================

\end{document}
