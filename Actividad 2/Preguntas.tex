\documentclass[11pt]{article}
\usepackage[spanish]{babel}
\usepackage[utf8]{inputenc}
\usepackage[T1]{fontenc}
\usepackage{graphicx}
\topmargin=-2.5cm
\textheight=24cm
\textwidth=16cm  
\oddsidemargin=0.45cm  
\setlength{\parindent}{0cm}
\renewcommand{\baselinestretch}{1.1}
\title{Preguntas de Reflexión. ACT2}
\author{Hinostroza Moya Natalia}
\date{01 de febrero de 2017}

%================================================================================
% INICIO
%================================================================================
\begin{document}

\textbf{\section*{\Large Preguntas de Reflexión}}

\begin{enumerate}
\item\large\textbf{¿Cúal es tu primera impresión del uso de bash/Emacs?}\\
Ya lo había utilizado antes, pero no para hacer lo que hicimos en esta actividad. Sin embargo, el ya haber usado un poco esta erramienta me hizo comenzar a trabajar esta vez con seguridad.

\item\large\textbf{¿Ya lo habías utilizado?}\\
sí.

\item\large\textbf{¿Qué cosas se te dificultaron más en bash/Emacs?}\\
Memorizar algunos comandos.

\item\large\textbf{¿Qué ventajas les ves a Emacs?}\\
Es capás de agilizar mucho el trabajo al trabajar con una cantidad grande de datos.

\item\large\textbf{¿Qué es lo que mas te llamó la atención en el desarrollo de esta actividad?}\\
El combinar el aprender Emacs con los sondeos.

\item\large\textbf{¿Qué cambiarías en esta actividad?}\\
Nada, es una buena práctica.

\item\large\textbf{¿Que consideras que falta en esta actividad?}\\
Nada, está bastante completa.

\item\large\textbf{¿Puedes compartir alguna referencia nueva que consideras util y no se haya contemplado?}\\
Todo me parecio bastante completo. La información dada en el laboratorio fue suficiente. 

\item\large\textbf{¿Algún comentario adicional que desees compartir?}\\
He funalizado esta practica, pero aún necesito reforzar más mi familiarización con Emacs. 


\vspace*{3cm}
\large Alumna: Hinostroza Moya Natalia\\
\large Profesor: Dr. Carlos Lizárraga 

\end{enumerate}

%================================================================================
% THE END
%================================================================================
\end{document}
